% mis paquetes
\usepackage[utf8]{inputenc} % Para codificar el archivo con UTF-8
\usepackage[english]{babel} % Para usar la traducción española
\usepackage[margin=1.5in]{geometry} % Para definir los márgenes del documento
\usepackage{amsmath} % Para utilizar simbolos matemáticos
\usepackage{amsfonts} % Para utilizar diferentes fuentes matemáticas
\usepackage{amsthm} % Para crear teoremas y definiciones matemáticas
\usepackage{amssymb} % Para utilizar símbolos matemáticos adicionales
\usepackage{graphicx} % Para incluir imágenes en el documento
\usepackage[usenames,dvipsnames]{xcolor} % Para utilizar diferentes colores
\usepackage{float} % Para definir flotantes (imágenes, tablas, etc.)
\usepackage[siunitx]{circuitikz} % Para crear diagramas de circuitos
\usepackage{tikz} % Para dibujar gráficos y diagramas
\usepackage{hyperref} % Para hacer enlaces hipervínculos en el documento
%\usepackage[numbers, square]{natbib} % Para crear bibliografías
\usepackage{fancybox} % Para crear cajas con bordes y fondos
\usepackage{epsfig} % Para incluir imágenes en formato EPS
\usepackage{soul} % Para resaltar texto con un fondo de color
\usepackage[framemethod=tikz]{mdframed} % Para crear marcos con bordes y fondos
\usepackage[colorinlistoftodos, color=orange!50]{todonotes} % Para agregar notas a tareas pendientes
\usepackage[framed,numbered,autolinebreaks,useliterate]{mcode} % Para incluir código en el documento
\usepackage[shortlabels]{enumitem} % Para personalizar listas enumeradas
\usepackage[version=4]{mhchem} % Para escribir fórmulas químicas
\usepackage{listings} % Para incluir código fuente en el documento
\usepackage{color} % Para personalizar los colores en el documento
\definecolor{mygreen}{RGB}{28,172,0} % Definir un color verde personalizado
\definecolor{mylilas}{RGB}{170,55,241} % Definir un color lila personalizado
\usepackage{subcaption} % Para crear subfiguras
\usepackage{url} % Para crear enlaces a páginas web
\usepackage{minted} % Para incluir código resaltado
\usepackage{wrapfig} % Para envolver texto alrededor de imágenes o figuras para crear una figura que "envuelve" el texto alrededor de ella.

\usepackage{siunitx}
\usepackage{vmargin}
\usepackage{layout}
\usepackage{fancyhdr}
\usepackage{vmargin}
\usepackage{setspace}
\usepackage{parskip}
\usepackage{caption} 
\usepackage[style=iso-numeric]{biblatex} %Bibliografia
\usepackage[final]{microtype}

\usepackage[style=iso-numeric]{biblatex}
\addbibresource{bibliography.bib}
\usepackage{tocloft}  % para list of equations

\newcommand{\listequationsname}{List of Equations}
\newlistof{myequations}{equ}{\listequationsname}
\newcommand{\myequations}[1]{%
    \addcontentsline{equ}{myequations}{\protect\numberline{\theequation}#1}\par}

\usepackage{bm} % bold math
\usepackage{acronym}
\usepackage[bottom]{footmisc}


%%%%%%%%%%%%%%%% DEFINICION DEL ESQUEMA DE COLORES DE PYTHON
%\usepackage{minted}
%\usepackage{xcolor} % to access the named colour LightGray
%\definecolor{LightGray}{gray}{0.9}
\usepackage{ulem} 
\usepackage{minted,xcolor}
\usemintedstyle{tango} %monokai
\definecolor{bg}{HTML}{E7E7E7} % from https://github.com/kevinsawicki/monokai

\lstset{
breakatwhitespace=false,	% sets if automatic breaks should only happen at whitespace
breaklines=true,	% sets automatic line breaking
frame=single,	% adds a frame around the code
numbers=left,	% where to put the line-numbers; possible values are (none, left, right)
showspaces=false,               % show spaces everywhere adding particular underscores; it overrides 'showstringspaces'
basicstyle=\small\ttfamily}
%\setlength{\parindent}{0cm}

%%%%%%%%%%%%%%%%%%%%%%%%%%%%%%%%%%%%%%%%%%%%%%%%%%%%%%%
%   _____ _    _  _____ _______ ____  __  __ 
%  / ____| |  | |/ ____|__   __/ __ \|  \/  |
% | |    | |  | | (___    | | | |  | | \  / |
% | |    | |  | |\___ \   | | | |  | | |\/| |
% | |____| |__| |____) |  | | | |__| | |  | |
%  \_____|\____/|_____/   |_|  \____/|_|  |_|
%%%%%%%%%%%%%%%%%%%%%%%%%%%%%%%%%%%%%%%%%%%%%%%%
%  _____ ____  __  __ __  __          _   _ _____   _____ 
% / ____/ __ \|  \/  |  \/  |   /\   | \ | |  __ \ / ____|
%| |   | |  | | \  / | \  / |  /  \  |  \| | |  | | (___  
%| |   | |  | | |\/| | |\/| | / /\ \ | . ` | |  | |\___ \ 
%| |___| |__| | |  | | |  | |/ ____ \| |\  | |__| |____) |
% \_____\____/|_|  |_|_|  |_/_/    \_\_| \_|_____/|_____/ 
%%%%%%%%%%%%%%%%%%%%%%%%%%%%%%%%%%%%%%%%%%%%%%%%%%%%%%%%%%

% SYNTAX FOR NEW COMMANDS:
%\newcommand{\new}{Old command or text}

% EXAMPLE:

\newcommand{\blah}{blah blah blah \dots}

\setlength{\parskip}{1em}

\renewcommand{\bibsection}{}

%#########################################################

%To use symbols for footnotes
\renewcommand*{\thefootnote}{\fnsymbol{footnote}}
%To change footnotes back to numbers uncomment the following line
%\renewcommand*{\thefootnote}{\arabic{footnote}}

% Enable this command to adjust line spacing for inline math equations.
% \everymath{\displaystyle}

% _______ _____ _______ _      ______ 
%|__   __|_   _|__   __| |    |  ____|
%   | |    | |    | |  | |    | |__   
%   | |    | |    | |  | |    |  __|  
%   | |   _| |_   | |  | |____| |____ 
%   |_|  |_____|  |_|  |______|______|
%%%%%%%%%%%%%%%%%%%%%%%%%%%%%%%%%%%%%%%



\bibliography{bibliography.bib}




